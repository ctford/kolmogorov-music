% Based on the template for sigplanconf LaTeX Class by Paul C. Anagnostopoulos (paul@windfall.com)
\documentclass[numbers]{sigplanconf}

\usepackage{amsmath}
\usepackage{hyperref}

\newcommand{\cL}{{\cal L}}

\begin{document}

\special{papersize=8.5in,11in}
\setlength{\pdfpageheight}{\paperheight}
\setlength{\pdfpagewidth}{\paperwidth}

\conferenceinfo{FARM 2016}{September 24, 2016, Nara, Japan}
\copyrightyear{2016}
%\copyrightdata{978-1-nnnn-nnnn-n/yy/mm}
%\copyrightdoi{nnnnnnn.nnnnnnn}

\title{Algomusicology, ???, Profit}
%\subtitle{Subtitle Text, if any}

\authorinfo{Chris Ford}
           {ThoughtWorks}
           {christophertford@gmail.com}

\maketitle

\begin{abstract}
    In this paper I propose a novel method for converting algomusicology into profit. While my particular example
    is slightly tongue-in-cheek, my serious point is that algorithmic composition can help us deconstruct and
    disrupt traditional notions of musical authorship.
\end{abstract}

%\category{CR-number}{subcategory}{third-level}

%\keywords
%keyword1, keyword2

\section{Introduction}

\section{Air on the {\textbackslash}G String}
Very great quantities of data can exhibit very little complexity. Consider the following Clojure expression, which
evaluates to a sequence of $10^{15}$ uppercase G characters:

\begin{verbatim}
(repeat (* 1000 1000 1000 1000 1000) \G)
\end{verbatim}

Though storing such a string would require roughly a petabyte of memory, representing it indirectly as a computation
takes very little space at all. To put it in perspective, that's roughly the size all music recorded to date,
described in less than a half a tweet \footnote{Based on the iTunes library featuring approximately 50 million songs
of approximately 4 megabytes each, and assuming it represents approximately a fifth of all recorded music.}.

We can use Clojure's macro facility to build a crude framework for comparing the size of descriptions and their
resulting values. The following two definitions are identical, save that description-length is a macro and therefore
operates on the unevaluated s-expression passed to it as an argument.

\begin{verbatim}
(defn result-length [expression]
  (-> expression print-str count))

(defmacro description-length [expression]
  (-> expression print-str count))

(fact "The result-length is how long the string
       representation of the evaluated result is."
  (result-length (repeat 65 \G)) => 131)

(fact "The description-length is how long the
       string representation of the expression is."
  (description-length (repeat 65 \G)) => 13)
\end{verbatim}

The randomness of a description can be measured by comparing the length of the textual description and the length
of the resulting value.

\begin{verbatim}
(defmacro randomness [expression]
  `(/ (description-length ~expression)
      (result-length ~expression)))

(fact "Kolmogorov randomness is the compression
       ratio between the description and the result."
  (randomness (repeat 65 \G)) => 13/131)
\end{verbatim}

\section{Analysis by compression}

David Meredith has used Kolmogorov complexity to compare the quality of musical analyses. The theory goes that the
more concise the algorithmic description of a piece, the better the musical model.

Using this idea, we can show that composing "Row, row, row your boat" out of sequences of durations and pitches
is a better description of the piece than the raw notes.

\begin{verbatim}
(def row-row
  "A simple melody built from durations and pitches."
              ; Row, row, row  your boat,
  (->> (phrase [3/3  3/3  2/3  1/3  3/3]
               [  0    0    0    1    2])
       (then
                ; Gent-ly   down the  stream,
         (phrase [2/3  1/3  2/3  1/3  6/3]
                 [  2    1    2    3    4]))
       (then    ; Merrily, merrily, merrily, merrily,
         (phrase (repeat 12 1/3)
                 (mapcat (partial repeat 3) [7 4 2 0])))
       (then
                ; Life is   but  a    dream!
         (phrase [2/3  1/3  2/3  1/3  6/3]
                 [  4    3    2    1    0]))
       (canon/canon (canon/simple 4))
       (where :pitch (comp scale/A scale/major))))
\end{verbatim}

Using a little more macro magic, we can calculate the ratio between the above description and the literal notes
of the song.

\begin{verbatim}
(defmacro definitionally [macro sym]
  (let [value (-> sym repl/source-fn read-string last)]
    `(~macro ~value)))

(fact "The definitionally macro lets us calculate on the
       definition of symbols."
  (definitionally description-length row-row) => 281
  (definitionally result-length row-row) => 2037
  (definitionally randomness row-row) => 281/2037)
\end{verbatim}

\section{The Library of Babel}

The Kleene star is the set of all possible substrings built from a set of elements.

\begin{verbatim}
(defn kleene* [elements]
  (letfn [(expand [strings]
            (for [s strings e elements] (conj s e)))]
    (->>
      (lazy-seq (kleene* elements))
      expand
      (cons []))))
\end{verbatim}

We can use the Kleene star with an alphabet of the ASCII alphabet to construct all possible ASCII strings as a lazy
sequence. This results in something very similar to the Library of Babel as imagined by Jorge Luis Borges. In this
vast building of library, are all possible books of four hundred and ten pages in length.

The difference is that our sequence has all possible strings of all finite lengths.

\begin{verbatim}
(defn library-of-babel []
  (let [ascii (->> (range 32 127) (map char))]
    (->> ascii
       kleene*
       (map (partial apply str)))))

(fact "We can construct all strings as a lazy
       sequence."
  (->> (library-of-babel) (take 5))
    => ["" " " "!" "\"" "#"]
  (nth (library-of-babel) 364645)
    => "GEB")
\end{verbatim}

Lexicon's are interesting from a Kolmogorov complexity perspective, because they can be described concisely yet contain
multitudes. Paradoxically, lexicons display the property that the whole is simpler than (almost all of) the parts. That is,
the Kolmogorov complexity of all strings in our Library of Babel beyond a certain size are more complicated to describe
than the Library as a whole.

Borges appreciated this paradox. His anonymous narrator concludes that "On some shelf in some hexagon (men reasoned)
there must exist a book which is the formula and perfect compendium of all the rest"?????. Any of the many books that contain
our library-of-babel function match the narrator's description, though the books that also contain the rest of this paper
must surely provide more helpful context.

\section{Contact}

The Champernowne word is a transcendental constant whose decimal expansion is the concatenation of the digits of the
natural numbers.

\begin{verbatim}
(defn decompose [n]
  (let [[remainder quotient] ((juxt mod quot) n 10)]
    (if (zero? quotient)
      [remainder]
      (conj (decompose quotient) remainder))))

(defn champernowne-word
  ([from]
   (->> (range)
        (map (partial + from))
        (mapcat decompose)))
  ([]
   (champernowne-word 0)))

(fact "The Champernowne word is defined by concatenating the
       natural numbers base 10."
  (->> (champernowne-word) (take 16))
    => [0 1 2 3 4 5 6 7 8 9 1 0 1 1 1 2])
\end{verbatim}

\section{Anti}

\section{Blurred Lines}

'Blurred Lines' is a pop song that was written by Pharrell Williams and Robin Thicke. In March 2015,
they were found to have infringed on the copyright of the Marvin Gaye song 'Got to Give It Up'.
\begin{verbatim}
(defn copyright-infringement-song
  ([skip-to]
   (->>
     (champernowne-word skip-to)
     (coding/decode 3)
     (where :time (bpm 120))
     (where :duration (bpm 120))))
   ([] (copyright-infringement-song 0)))

(def blurred-lines 12450012001200311273127612731276127312761273127612731276127312761245001200121245001200120031127312761273127612731276127312761273127612731276124500120012124500120012003112731276127312761273127612731276127312761273127612450012001212450012001200311273127612731276127312761273127612731276127312761245001200121240001200120031126812711268127112681271126812711268127112681271124000120012124000120012003112681271126812711268127112681271126812711268127112400012001212400012001200311268127112681271126812711268127112681271126812711240001200121252004100411264125012621249126112471245)
\end{verbatim}

\bibliographystyle{plainnat}

% The bibliography should be embedded for final submission.

\begin{thebibliography}{}
\softraggedright

\bibitem{Overtone} Aaron, Sam and Rose, Jeff: Overtone (\url{http://overtone.github.io})
\bibitem{Anti EP} Autechre: Anti EP (\url{https://en.wikipedia.org/wiki/Anti_EP})

\end{thebibliography}

\end{document}
